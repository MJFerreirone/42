% This is samplepaper.tex, a sample chapter demonstrating the
% LLNCS macro package for Springer Computer Science proceedings;
% Version 2.20 of 2017/10/04
%
\documentclass[runningheads]{llncs}
%
\usepackage{graphicx}
\usepackage{todonotes}
% Used for displaying a sample figure. If possible, figure files should
% be included in EPS format.
%
% If you use the hyperref package, please uncomment the following line
% to display URLs in blue roman font according to Springer's eBook style:
% \renewcommand\UrlFont{\color{blue}\rmfamily}

\usepackage[backend=biber,sorting=nyt,bibencoding=latin1,abbreviate=false,mincrossrefs=3,style=numeric,maxnames=30]{biblatex}
\usepackage{multirow}
 \usepackage{graphicx}
\addbibresource{bib/db.bib}


\begin{document}
%
\title{Learning Class Definitions Using Ontology Embeddings\thanks{Technical report of the task force 42 from ISWS 2022 led by Heiko Paulheim.}}

%
%\titlerunning{Abbreviated paper title}
% If the paper title is too long for the running head, you can set
% an abbreviated paper title here
%
\author{First Author\inst{1}\orcidID{0000-1111-2222-3333} \and
Second Author\inst{2,3}\orcidID{1111-2222-3333-4444} \and
Third Author\inst{3}\orcidID{2222--3333-4444-5555}}
%
\authorrunning{F. Author et al.}
% First names are abbreviated in the running head.
% If there are more than two authors, 'et al.' is used.
%
\institute{Princeton University, Princeton NJ 08544, USA \and
Springer Heidelberg, Tiergartenstr. 17, 69121 Heidelberg, Germany
\email{lncs@springer.com}\\
\url{http://www.springer.com/gp/computer-science/lncs} \and
ABC Institute, Rupert-Karls-University Heidelberg, Heidelberg, Germany\\
\email{\{abc,lncs\}@uni-heidelberg.de}}
%
\maketitle              % typeset the header of the contribution
%
\begin{abstract}
The abstract should briefly summarize the contents of the paper in
15--250 words.  Test citation \cite{Berners-Lee2001}

\keywords{Ontology Embeddings \and  Class Definition \and  DBpedia.}
\end{abstract}

\section{Research Questions}
\label{sec:rq}
Are knowledge graph embeddings helpful in predicting entity classes?


\section{Empirical Semantics}
\label{sec:def}
The empirical semantics perspective of this work considers classes and properties of the DBpedia ontology. The main problematic aspect is the semantic expressivity and formalisation of the DBpedia categories (\textit{skos:Concept}) derived from Wikipedia.

\section{Introduction}
\label{sec:intro}
\textit{[1 Page]}

\noindent
Explain your perspective on the problem of Empirical Semantics. 
Give both the intuition and motivate, by relying on use cases and examples, why this perspective is important. 
Briefly describe what is the state of the art and how you’re pushing it with your contribution. 
Also mention what data and methods you use in your work. 
Conclude by clearly stating what is your contribution.

Wikipedia categories serve as topic definitions of resources in DBpedia.


\section{Related Work}
\label{sec:related}
\textit{[1 Page]}

The objective of enriching knowledge graphs such as DBpedia, has been attracting the efforts of the scientific community during the last years, and continues in the same direction still nowadays. As this work, in order to attack the issue, highlights the path of exploiting the categories, some of the main related work is presented as follows:
\begin{itemize}
  \item Uncovering the Semantics of Wikipedia Categories. \cite{HeistHeiko2019}
  This work introduces an approach for the discovery of category axioms that uses information from the category network, category instances, and their lexicalisations with DBpedia as background knowledge.
\end{itemize}

\noindent
List the main relevant work (a bullet list is ok) and for each of them write a paragraph describing (i) the key contribution of the related work, (ii) how your contribution relates/differentiate from it.


\section{Resources}
\label{sec:resources}
\textit{[1 page]}

\noindent
List here what datasets are you using and why.

\section{Proposed approach}
\label{sec:approach}
\textit{[2 pages]}

\noindent
Describe your proposed method. 


\section{Evaluation and Results: Use case/Proof of concept - Experiments}
\label{sec:evaluation}
\textit{[2 pages]}

\noindent
Show here that your proposed approach addresses your research questions or how you intend to show it. This can be done by either or both:

\begin{itemize}
    \item Describing an experimental setting design, including research hypotheses, methods and metrics of measurements 

    \item Describing a proof of concept/use case, based on real data, that support your claim
\end{itemize}



\section{Discussion and Conclusions}
\label{sec:conclusions}
\textit{[1 page]}

\noindent
Identify strengths and weaknesses of your proposal, discuss lessons learned: what are the key issues you have encountered or that you think should be taken into account to develop your proposal/experiments, and what are possible ways to address them. 





\printbibliography
\end{document}